\begin{document}
%%% {{{ Title  

\title{Mathematics for computer generated spoken documents.}
\author{T. V. Raman\\
Department of Computer Science\\
4116 Upson Hall\\
Cornell University\\
Ithaca NY 14853--7501\\
\voicemail{(607)255-9202}\\
\email{raman@cs.cornell.edu}}
\date{May 27, 1992}
\maketitle
\begin{abstract}
  This document contains the math examples for testing   the audio
  formatting rules when developing Aster.
\end{abstract}
%%% }}}
%%% {{{ simple fractions and expressions

\section{simple fractions and expressions. }

%%
$$a+b+c+d$$



%%
$$a+\frac{b}{c} +d$$


%%
$$\frac{a+b}{c+d}$$


%%
$$\frac{a}{b}+c+d$$


%%
$$\frac{a}{b+c+d}$$


%%
$$a+\frac{b+c}{d+e}+x$$


%%
$$a+bc+d$$


%%
$$(a+b)(c+d)$$

%%% }}}
%%% {{{ superscripts and subscripts.

\section{superscripts and subscripts.  }


%%
$$x^k_1 +x^k_2 + x^k_3 + \cdots + x^k_n = 0$$

%%
$$x^{k_1} + x^{k_2} + x^{k_3} + \cdots + x^{k_n} = 0$$

%%
$$x_{k^1}+x_{k^2}+x_{k^3}+\cdots+x_{k^n}=0$$

%%
$$x^{k^1}+x^{k^2}+x^{k^3}+\cdots+x^{k^n}=0$$

%%
$$x_{k_1}+x_{k_2}+x_{k_3}+\cdots+x_{k_n}=0$$

%%
$$x +_n y +_n z$$

%%% }}}
%%% {{{ Knuth's examples of fractions and exponents

\section{Knuth's examples of fractions and exponents. }


%%
$$x+y^2\over k+1$$

%%
$${x+y^2\over k}+1$$

%%
$$x+{y^2\over k+1}$$

%%
$$x+{y^2\over k}+1$$

%%
$$x+y^{2\over k+1}$$

%%
$$x^{2^y} \neq {x^2}^y$$

%%
$${x^2}^y = x^{2y}$$

%%

%%% }}}
%%% {{{ Continued fraction

\section{A continued fraction. }

\[ 
1+  {x \over
 {\scriptstyle 1+ {\scriptstyle x \over
 {\scriptstyle 1 + {\scriptstyle x \over
 {\scriptstyle 1 + {\scriptstyle x \over
 {\scriptstyle 1+ {\scriptstyle x \over
1+{\scriptstyle
  \atop \ddots }}}}}}}}}}
\]

%%

%%% }}}
%%% {{{ Simple  School  algebra.

\section{Simple  School  algebra.   }

$$(a+b)^3=a^3+3a^2b+3ab^2+b^3$$

%%
$$a^3+b^3=(a+b)(a^2-ab+b^2)$$

%%
Given $ax^2+bx+c=0$, we have
$$x= \frac{-b \pm \sqrt{b^2-4ac}}{2a}$$

%%

%%% }}}
%%% {{{ square roots.

\section{square roots.  }

$$\frac{1+\sqrt{5}}{2}=\phi$$

%%
$$\frac{\sqrt{\pi}}{2} \neq \sqrt{\frac{\pi}{2}}$$

%%
$$\sqrt{1+\sqrt{2+\sqrt{2+\sqrt {2+\cdots }}}}$$

%%

%%% }}}
%%% {{{ Trigonometric identities

\section{Trigonometric identities. }

$$\sin^2x+\cos^2x=1$$

%%
$$\sin x^2 + \cos x^2 \neq 1 $$

%%
$$\sin^{-1}x \neq \sin x^{-1}$$

%%
$$\sin (a+b) = \sin a \cos b + \cos a \sin b$$

%%
$$\cos (x+y)=\cos x \cos y - \sin x \sin y$$

%%
$$\sin 2x = 2 \sin x \cos x $$

%%
$$\cos 2x = \cos^2 x -\sin^2 x$$

%%

%%% }}}
%%% {{{ logs

\section{Logarithms. }

$$\log^2x\neq2\log x$$

%%
$$\log x^2=2\log x$$

%%
$$\frac{\log x}{\log a} = \log_a x$$

%%
$$\log_{a^2} x = \frac{1}{\log_x a^2}= \frac{1}{2\log_x a} =
\frac{\log_a x}{2}$$ 

%%

%%% }}}
%%% {{{ Series

\section{Series. }

$$1+x+x^2+x^3+x^4+\cdots+x^{n-1}+\cdots  = \frac{1}{1-x}$$

%%
$$1+x+\frac{x^2}{2}+\frac{x^3}{3}+\cdots +\frac{x^n}{n}+\cdots$$

%%
$$  x - \frac{x^2}{2} +\frac{x^3}{3}
-\frac{x^4}{4}+\frac{x^5}{5} \pm \cdots  = \log(1+x)$$

%%
$$\gamma = 1+\frac{1}{2}+\frac{1}{3} +\frac{1}{4} +\cdots +\frac{1}{n}
-\log n $$

%%
$$\log (1+x) - \log (1-x) = \log \frac{1+x}{1-x} = \sum_{i=1}^\infty
\frac{x^{2i-1}}{2i -1}$$

%%% }}}
%%% {{{ Integrals

\section{Integrals. }


%%
$$\int\frac{\dx}{x} =\log x$$

$$\int_1^a \int_1^b\int_1^c e^{x+y+z}\dx\dy\dz$$

%%
$$\int_1^\infty e^{x^2-x-1}\dx$$

%%
$$\int_1^\infty e^{x^{2-x}-1}\dx$$

%%
$$\int_0^1\int_0^{\sqrt{1 -y^2}}1\dx\dy= \int_0^{\pi/2}\int_0^1
r\varint{r}\varint{\theta}$$

$$s=\int_a \int_b f \dx\dy+1$$

%%

%%% }}}
%%% {{{ Summation

\section{Summations. }

$$\sum_{i=1}^n a_i =1$$

%%
$$\sum_{1\leq i\leq n}a_i =1$$

%%
$$\sum_{i=1}^n a_i + b_i = 1$$

%%

%%% }}}
%%% {{{ Limits

\section{Limits. }

$$\lim_{x \to \infty}\int_0^x e^{-y^2}\dy = \frac{\sqrt{\pi}}{2}$$

%%
$$\lim_{x\to 0}\frac{\sin x}{x} =1$$

%%

%%% }}}
%%% {{{ Cross referenced equations

\section{Cross referenced equations. }

\begin{equation}
\cosh x = \frac{e^x + e^{-x} }{2} \label{eq:cosh}
\end{equation}

\begin{equation}
\sinh x = \frac{e^x-e^{-x}}{2} \label{eq:sinh}
\end{equation}

Squaring~\ref{eq:cosh} and~\ref{eq:sinh} and computing their
difference gives
$$\cosh^2x -\sinh^2 x = 1$$

%%

%%% }}}
%%% {{{ Distance formula

\section{Distance formula. }


%%
Given $x=(x_1,x_2), y=(y_1,y_2)$ the distance between the two points
is given by: 
$$d(x,y) = \sqrt{(x_1-y_1)^2 +(x_2-y_2)^2} $$
This is the distance formula.

%%% }}}
%%% {{{ Quantified expression

\section{Quantified expression. }


%%
$$\forall x \in X:    \exists y \in Y :   x=y$$

%%% }}}
%%% {{{ Exponentiation

\section{Exponentiation}
Consider the expression:
$$e^{e^{e^x}}$$
Differentiating with respect to $x$ gives:
$$  e^{e^{e^x}} e^{e^x} e^x$$
Simplifying this expression gives:
$$  e^{(e^{e^x} + e^x + x)}$$

%%% }}}
%%% {{{Matrix

\section{A generic matrix}
Notice the use of vertical and diagonal dots in the generic matrix
shown below.
$$A=\pmatrix{a_{1 1}&a_{1 2}&\ldots&a_{1 n}\cr
             a_{2 1}&a_{2 2}&\ldots&a_{2 n}\cr
             \vdots&\vdots&\ddots&\vdots\cr
             a_{m 1}&a_{m 2}&\ldots&a_{m n}\cr}$$

%%% }}}
%%% {{{ Faa de Bruno's formula

\section{Faa de Bruno's formula }


%%
Let $D^k_xu$ represent the $k$th derivative of a function $u$ with
respect to $x$.  The chain rule states that $D^1_xw = D^1_uw
D^1_xu$.  If we apply this to second derivatives, we find $D^2_xw =
D^2_uw (D^1_xu)^2+D^1_uw D^2_xu$.  Show that the {\em general formula}
is

%%
\begin{equation}\label{eq:faa-de-bruno}
D^n_xw =
\sum_{0\le j\le n}
\sum_{\scriptstyle k_1+k_2+\cdots+k_n=j
\atop {\scriptstyle k_1+2k_2+\cdots+nk_n=n
\atop  {\scriptstyle k_1,k_2,\ldots,k_n\ge0
}}}
D^j_u w \frac{n! 
{(D^1_x u)}^{k_1}
\cdots {(D^n_x u)}^{k_n}
}
{k_1!{(1!)}^{k_1} \cdots k_n!{(n!)}^{k_n}} 
\end{equation}
%%

%%% }}}

\end{document}
