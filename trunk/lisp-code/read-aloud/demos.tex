		% Document Type: LaTeX
% Master File: demos.tex
\input {headers}
%%% {{{ old demos.

%these have been filtered out to extended demos

%%
$$a+b+c+d$$
%%
$$a+\frac{b}{c} +d$$
%%
$$\frac{a+b}{c+d}$$
%%
$$\frac{a}{b}+c+d$$
%%
$$\frac{a}{b+c+d}$$
%%
$$a+b+\frac{c}{d}$$
%%
$$\frac{a+b+c}{d}$$
%%
$$a+\frac{b+c}{d}$$
%%
$$a+\frac{b+c}{d+e}+x$$
%%
$$a+\frac{b}{c}+x$$
%%
$$a+bc+d$$
%%
$$(a+b)(c+d)$$
%%
$$ab+c+d$$
%%
$$a+b+cd$$
%%
$$a+b/c+d$$
%%
$$(a+b)/(c+d)$$
%%
$$a/b+c+d$$
%%
$$a+b+c/d$$
%%
$$(a+b+c)/d$$
%%
$$a+(b+c)/d$$
%%
$$a+(b+c)/d +e$$
%%
$$a+(b+c)/(d+e)$$
%%
$$a-(b+c)/d$$
%%
$$\frac{a-(b+c)}{d}$$
%%
$$a-(b+c)/d=1$$
%%
$$a-(b-c)/d$$
%%
$$a-(b-c)d$$
%%
$$\frac{a-(b-c)}{d}$$
%%
$$x+y^2\over k+1$$
%%
$${x+y^2\over k}+1$$
%%
$$x+{y^2\over k}+1$$
%%
$$x+{y^2\over k+1}$$
%%
$$x+y^{2\over k+1}$$
%%
$$1+  {x \over {1+ {x \over {1 + {x \over {1 + {x \over {1+ {x \over
1+\cdots }}}}}}}}}$$
%%
$$(a+b)^3=a^3+3a^2b+3ab^2+b^3$$
%%
$$a^3+b^3=(a+b)(a^2-ab+b^2)$$
%%
Given $ax^2+bx+c=0$, we have
$$x= \frac{-b \pm \sqrt{b^2-4ac}}{2a}$$
%%
$$\frac{1+\sqrt{5}}{2}=\phi$$
%%
$$\int\frac{\dx}{x} =\log x$$
%%
$$\frac{\sqrt{\pi}}{2} \neq \sqrt{\frac{\pi}{2}}$$
%%
$$\sqrt{1+\sqrt{2+\sqrt{2+\sqrt {2+\cdots }}}}$$
%%
$$\sin^2x+\cos^2x=1$$
%%
$$\sin x^2 + \cos x^2 \neq 1 $$
%%
$$\sin^{-1}x \neq \sin x^{-1}$$
%%
$$\log^2x\neq2\log x$$
%%
$$\log x^2=2\log x$$
%%
$$\frac{\log x}{ \log a} = \log_a x$$
%%
$$\log_{a^2} x = \frac{1}{\log_x a^2}= \frac{1}{2\log_x a} =
\frac{\log_a x}{2}$$ 
%%
$$1+x+x^2+x^3+x^4+\cdots+x^{n-1}+\cdots  = \frac{1}{1-x}$$
%%
$$1+x+\frac{x^2}{2}+\frac{x^3}{3}+\cdots +\frac{x^n}{n}+\cdots$$
%%
$$  x - \frac{x^2}{2} +\frac{x^3}{3}
-\frac{x^4}{4}+\frac{x^5}{5} +\cdots  = \log(1+x)$$
%%
$$\gamma = 1+\frac{1}{2}+\frac{1}{3} +\frac{1}{4} +\cdots +\frac{1}{n}
-\log n $$
%%
$$\int_1^a \int_1^b\int_1^c e^{x+y+z}\dx\dy\dz$$
%%
$$\int_1^\infty e^{x^2-x-1}\dx$$
%%
$$\int_1^\infty e^{x^{2-x}-1}\dx$$
%%
$$\int_0^1\int_0^{\sqrt{1 -y^2}}1\dx\dy= \int_0^{\pi/2}\int_0^1
r\varint{r}\varint{\theta}$$
%%
$$-\int_1^a \int_1^b\int_1^c e^{x+y+z}\dx\dy\dz=1$$
%%
$$s=\int_a \int_b f \dx\dy+1$$
%%
$$\sum_{i=1}^n a_i =1$$
%%
$$\sum_{1\leq i\leq n}a_i =1$$
%%
$$\sum_{i=1}^n a_i + b_i = 1$$
%%
$$\sum_{i=1}^n a_i b_i c_i = 1$$
%%
$$\lim_{x \to \infty}\int_0^x e^{-y^2}\dy = \frac{\sqrt{\pi}}{2}$$
%%
$$\lim_{x \to \infty}\{\int_0^x e^{-t^2}\dt\} = \frac{\sqrt{\pi}}{2}$$
%%
$$\sin (a+b) = \sin a \cos b + \cos a \sin b$$
%%
$$\cos (x+y)=\cos x \cos y - \sin x \sin y$$
%%
$$x^{2^y} \neq {x^2}^y$$
%%
$${x^2}^y = x^{2y}$$
%%
$$x\sin x +y\sin y + z\sin z$$
%%
$$x\sin x\cos y\tan z$$
%%
$$\sin 2x = 2 \sin x \cos x $$
%%
$$\cos 2x = \cos^2 x -\sin^2 x$$
%%
\begin{equation}
\cosh x = \frac{e^x + e^{-x} }{2} \label{eq:cosh}
\end{equation}

\begin{equation}
\sinh x = \frac{e^x-e^{-x}}{2} \label{eq:sinh}
\end{equation}
Squaring  \ref{eq:cosh} and \ref{eq:sinh} and computing their
difference gives
$$\cosh^2x -\sinh^2 x = 1$$
%%
$$\log (1+x) - \log (1-x) = \log \frac{1+x}{1-x} = \sum_{i=1}^\infty
\frac{x^{2i-1}}{2i -1}$$
%%
$$\sum_{1 \leq i \leq n} \int_1^a f\dx = \int_1^a \sum_{1\leq i \leq n
} f \dx $$
%%
$$\tan (a+b) =\frac{\tan a + \tan b}{  1 - \tan a \tan b}$$
%%
$$\lim_{x\to0}\lim_{y\to\infty}\int_x^ye^t\dt$$
%%
$$\int_0^\infty e^{-(1+t^2)} \dt = \frac{\sqrt{\pi}}{2e}$$
%%
$$\bigcup_{x_i \in X} A\cap x_i = \emptyset$$.
%%
$$T\succ_\alpha T_1 \succ_\alpha T_2 \succ_\alpha \cdots$$
%%
$$x\sin(x+y\cos (y+z\tan z))$$
%%
$$x +_n y +_n z$$
%%
$$x_1 +_n x_2 +_n x_3 +_n \cdots +_n x_m$$
%%
$$\forall x \in X:    \exists y \in Y :   x=y$$
%%
Given $x=(x_1,x_2), y=(y_1,y_2)$ the distance between the two points
is given by: 
$$d(x,y) = \sqrt{(x_1-y_1)^2 +(x_2-y_2)^2} $$
This is the distance formula.
%%
$$x^k_1 +x^k_2 + x^k_3 + \cdots + x^k_n = 0$$.
%%
$$x^{k_1} + x^{k_2} + x^{k_3} + \cdots + x^{k_n} = 0$$.
%%
$$x_{k^1}+x_{k^2}+x_{k^3}+\cdots+x_{k^n}=0$$.
%%
$$x^{k^1}+x^{k^2}+x^{k^3}+\cdots+x^{k^n}=0$$.
%%
$$x_{k_1}+x_{k_2}+x_{k_3}+\cdots+x_{k_n}=0$$.
%%
$${x^k}^1 +{x^k}^2 +{x^k}^3 + \cdots + {x^k}^n = 0$$
%%
$$x^{k_1^l+k_2^l+k_3^l+\cdots+k_n^l}_{j_{1_i}+j_{2_i}+j_{3_i}+\cdots+j_{m_i}}$$.
%%
$$x^{k_1^l+k_2^l+k_3^l+\cdots+k_n^l}_{j_1^i+j_2^i+j_3^i+\cdots+j_m^i}$$. 
%%
Evaluate
$$\int_1^a \int_1^b\int_1^c e^{x+y+z}\dx\dy\dz$$
%%
Solution:
Evaluating the inner most integral we get:
$$\int_1^a \int_1^b[e^{x+y+z}]_{x=1}^{x=c}\dy\dz$$
%%
Taking limits gives:
$$\int_1^a \int_1^be^{c+y+z}-e^{1+y+z}\dy\dz$$
%%
Simplifying the above gives:
$$(e^c-e)\int_1^a\int_1^b e^{y+z}\dy\dz$$
%%
Continuing as above we get:
$$(e^c-e)(e^b-e)(e^a-e)$$
%%

\[
\begin{eqnarray}
\int_1^a \int_1^b\int_1^c e^{x+y+z}\dx\dy\dz
&=& \int_1^a \int_1^b[e^{x+y+z}]_{x=1}^{x=c}\dy\dz\\
&=& \int_1^a \int_1^be^{c+y+z}-e^{1+y+z}\dy\dz \\
&=& (e^c-e)\int_1^a\int_1^b e^{y+z}\dy\dz \\
&=& (e^c-e)\int_1^a [e^{y+z}]_{y=1}^{y=b}\dz \\
&=& (e^c -e) (e^b -e) \int_1^a e^z\dz \\
&=&  (e^c-e)(e^b -e)[e^z]_{z=1}^{z=a}\\
&=& (e^c -e)(e^b -e )(e^a -e)
\end{eqnarray}
\]
%%
$$\frac{1}{\Gamma(s)} =
s e^{\gamma s} \prod_{n=1}^{\infty} \left\{ \left( 1 +
\frac{s}{n}\right) e^{-s/n}\right\}$$
%%
Above with the arguments to the continuous product re-ordered:
$$\frac{1}{\Gamma(s)} =
s e^{\gamma s} \prod_{n=1}^{\infty} \left\{ e^{-s/n} \left( 1 +
\frac{s}{n}\right) \right\}$$
%%
%%
Let $D^k_xu$ represent the $k$th derivative of a function $u$ with
respect to $x$.  The chain rule states that $D^1_xw = D^1_uw
D^1_xu$.  If we apply this to second derivatives, we find $D^2_xw =
D^2_uw (D^1_xu)^2+D^1_uw D^2_xu$.  Show that the {\em general formula}
is
%%
$$D^n_xw =
\sum_{0\le j\le n}
\sum_{\scriptstyle k_1+k_2+\cdots+k_n=j
\atop {\scriptstyle k_1+2k_2+\cdots+nk_n=n
\atop  {\scriptstyle k_1,k_2,\ldots,k_n\ge0
}}}
D^j_u w \frac{n! 
{(D^1_x u)}^{k_1}
\cdots {(D^n_x u)}^{k_n}
}
{k_1!{(1!)}^{k_1} \cdots k_n!{(n!)}^{k_n}} $$
%%

\[
x^{k_1^l+k_2^l+k_3^l+\cdots+k_n^l}_{j_{1^i}+j_{2^i}+j_{3^i}+\cdots+j_{m^i}}
\]
$$x^{{k_1^l+k_2^l+k_3^l+\cdots+k_n^l}_{j_1^i+j_2^i+j_3^i+\cdots+j_m^i}}$$. 
%%
$$x_{k_i} +{x_k}_i+{x_k}i +x^{k^i} + {x^k}^i+{x^k}i =0$$
%%

%%% }}}

%%% {{{ new demos.

%Sat Feb  6 19:31:50 EST 1993
%these demos are being added here, will move evenetually to extended
%demos after getting filtered. 

$x+y^{\frac{2}{k}+1}$
$x+y^{\frac{2+i}{k}+1}$
$x+y^{\frac{2+i}{k+i}+1}$
$x+y^{\frac{2+i}{k+i}}+1$

%Sun Feb  7 19:33:48 EST 1993
%taken from emacs calc:
%got by differentiating and simplifying e^e^..
$$  e^{(e^{e^{e^x}} + e^{e^x} + e^x + x)} 
      (e^{e^{e^x}} e^{e^x} e^x + e^{e^x} e^x + e^x + 1)$$

$$  e^{(e^{(e^{(e^{(e^x + 1)} + 1)} + 1)} + 1)}$$
differentiating with $x$ gives:
$$  e^{(e^{(e^{(e^{(e^x + 1)} + 1)} + 1)} + 1)} 
      e^{(e^{(e^{(e^x + 1)} + 1)} + 1)} e^{(e^{(e^x + 1)} + 1)} e^{(e^x + 1)} 
      e^x$$

Simplifying above gives:
$$  e^{(e^{(e^{(e^{(e^x + 1)} + 1)} + 1)} + e^{(e^{(e^x + 1)} + 1)} 
          + e^{(e^x + 1)} + e^x + x + 4)} $$

$$ (e^{(e^{e^{e^x}} + e^{e^x} + e^x + x)} + e^{(e^{e^x} + e^x + x)} 
       + e^{(e^x + x)} + e^x + 1) 
      e^{(e^{e^{e^{e^x}}} + e^{e^{e^x}} + e^{e^x} + e^x + x)}$$

$$1+e^{\pi i}=0$$

$$e^{\pi i} + 1 = 0$$
%%% }}}



